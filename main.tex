\documentclass[12pt,a4paper]{article}

% Language support for Greek and English
\usepackage[a4paper, margin=1in]{geometry}
\usepackage[greek,english]{babel}
\usepackage[utf8]{inputenc}
\usepackage[T1]{fontenc}
\usepackage{lmodern}
\usepackage{graphicx}
\usepackage{placeins}

% Begin the document
\begin{document}
\selectlanguage{greek}

% Title and author information
\title{Επιστημονικός Υπολογισμός 2023-24}
\author{ΣΟΥΛΙ ΣΠΥΡΟ\\Α.Μ. 1070263}
\date{Πάτρα 2023/01/10}
\maketitle

% Table of Contents
\tableofcontents
\newpage
\selectlanguage{greek}
% Start of content
\section{Στοιχεία υπολογιστικού συστήματος}
\selectlanguage{english}
% Inserting the table
\begin{table}[h]
\centering
\begin{tabular}{|l|l|}
\hline
\textbf{\selectlanguage{greek}Χαρακτηριστικά} &\selectlanguage{greek} \textbf{Απάντηση} \\
\hline
\selectlanguage{greek} Έναρξη/λήξη εργασίας & 1/9/2024 – 1/13/2024 \\
model & ASUS G531GU \\
O/S & Windows 10 Pro \\
processor name & 4-Core Intel Core i5 9300H \\
processor speed & 3.9 GHz \\
number of processors & 1 \\
total \# cores & 4 \\
total \# threads & 8 \\
FMA instruction & yes \\
L1 cache & 128KB Instruction, 128KB Data write-back \\
L2 cache & 128 write-back \\
L3 cache & (shared)8MB write-back \\
Gflops/s & 301.0 \\
Memory & 16GB \\
Memory Bandwidth & 40.5 Gb/s \\
MATLAB Version & (9.14.0.2206163) R2023a Last Update February 22, 2023 \\
BLAS & Intel(R) oneAPI Math Kernel Library Version 2021.3 \\
LAPACK & Intel(R) oneAPI Math Kernel Library Version 2021.3 \\
& supporting Linear Algebra PACKage (LAPACK 3.9.0) \\
\hline
\end{tabular}
\caption{\selectlanguage{greek}Στοιχεία υπολογιστικού συστήματος}
\label{tab:system_info}
\end{table}

% Inserting the first image
\begin{figure}[h]
\centering
\includegraphics[width=1\textwidth]{bench1.png} % Replace 'image1.jpg' with the actual filename
\label{fig:image1}
\end{figure}

% Inserting the second image
\begin{figure}[h]
\centering
\includegraphics[width=1\textwidth]{bench.png} % Replace 'image2.jpg' with the actual filename
\label{fig:image2}
\end{figure}
\FloatBarrier % Prevents floats from moving past this point

\section{\selectlanguage{greek}Χρονομετρήσεις}
\subsection{}
\begin{figure}[h]
\centering
\includegraphics[width=1\textwidth]{image2.png} % Replace 'image2.jpg' with the actual filename
\label{fig:image3}
\end{figure}
\begin{figure}[h]
\centering
\includegraphics[width=1\textwidth]{image3.png} % Replace 'image2.jpg' with the actual filename
\label{fig:image4}
\end{figure}
\FloatBarrier % Prevents floats from moving past this point
\selectlanguage{greek}
Σύντομη περιγραφή όσων έκανα και σχολιασμός των αποτελεσμάτων:
Παραγωγή δεδομένων: Δημιουργήσαμε δεδομένα εκτελώντας τη λειτουργία αποσύνθεσης \selectlanguage{english}Cholesky\selectlanguage{greek} για πίνακες μεγεθών \selectlanguage{english}$n = 100 \times 2^{[0:6]}$
. \selectlanguage{greek}Για κάθε μέγεθος πίνακα, μετρήσαμε το χρόνο εκτέλεσης χρησιμοποιώντας τη συνάρτηση \selectlanguage{english}timeit.
\selectlanguage{greek}
Προσαρμογή κυβικής συνάρτησης: Χρησιμοποιήσαμε τη συνάρτηση\selectlanguage{english} polyfit \selectlanguage{greek}για να προσαρμόσουμε μια κυβική συνάρτηση στα παραγόμενα δεδομένα. Ενσωματώσαμε τη συμβουλή να χρησιμοποιούμε βάρη \selectlanguage{english}(mu) \selectlanguage{greek}για πιο αξιόπιστα αποτελέσματα.
Αποτελέσματα και συντελεστές: Εξήχθησαν οι συντελεστές της προσαρμοσμένης κυβικής συνάρτησης\selectlanguage{english} $(alpha3n^3 + alpha2n^2 + alpha1n + alpha0)$.\selectlanguage{greek} Οι συντελεστές παρέχουν πληροφορίες σχετικά με τους ρυθμούς ανάπτυξης και τις σταθερές που συμμετέχουν στην κυβική συνάρτηση.
Απεικόνιση: Δημιουργήσαμε μια γραφική παράσταση για την απεικόνιση των μετρούμενων δεδομένων και της προσαρμοσμένης κυβικής συνάρτησης.
Σχολιασμός των αποτελεσμάτων: Η συμπερίληψη βαρών στη συνάρτηση\selectlanguage{english} polyfit\selectlanguage{greek} με την παράμετρο mu ενισχύει την αξιοπιστία της διαδικασίας προσαρμογής. Λαμβάνει υπόψη την κλίμακα των δεδομένων και την κεντράρει κατάλληλα, οδηγώντας σε ακριβέστερες εκτιμήσεις των συντελεστών. Η καμπύλη που σχεδιάζεται παρέχει μια οπτική αναπαράσταση του πόσο καλά η κυβική συνάρτηση προσαρμόζεται στα μετρούμενα δεδομένα.


% Insert your content here
\subsection{}
\begin{figure}[h]
\centering
\includegraphics[width=1\textwidth]{image4.png} 
\label{fig:image5}
\end{figure}
\begin{figure}[h]
\centering
\includegraphics[width=1\textwidth]{image5.png} % 
\label{fig:image6}
\end{figure}
\selectlanguage{greek}
Σύντομη περιγραφή όσων έκανα και σχολιασμός των αποτελεσμάτων:\\

Εξαγωγή συντελεστών: Οι συντελεστές \selectlanguage{english} alpha3, alpha2, alpha1 alpha0\selectlanguage{greek} προέκυψαν από την πράξη\selectlanguage{english} polyfit.\selectlanguage{greek}
Προβλεπόμενοι χρόνοι εκτέλεσης: Η λειτουργία polyval χρησιμοποιήθηκε για τον υπολογισμό των προβλεπόμενων χρόνων εκτέλεσης για δύο σύνολα τιμών\selectlanguage{english} n:
[100:100:2000]:\selectlanguage{greek} Ένα εύρος που συνάδει με τα αρχικά δεδομένα.
[150:100:1550]: Ένα διαφορετικό υποσύνολο τιμών\selectlanguage{english} n\selectlanguage{greek} για πρόσθετες προβλέψεις.
Εμφάνιση των αποτελεσμάτων: Εμφανίστηκαν οι προβλεπόμενοι χρόνοι εκτέλεσης και για τα δύο σύνολα τιμών \selectlanguage{english}n.\selectlanguage{greek}\\

Σχολιασμός των αποτελεσμάτων: Οι προβλεπόμενοι χρόνοι εκτέλεσης παρέχουν εκτιμήσεις με βάση την κυβική συνάρτηση που λαμβάνεται από τα μετρούμενα δεδομένα. Το πρώτο σύνολο τιμών\selectlanguage{english} n ([100:100:2000]) \selectlanguage{greek}αντιστοιχεί στο αρχικό εύρος που χρησιμοποιήθηκε για την προσαρμογή της κυβικής συνάρτησης. Οι προβλέψεις για αυτό το εύρος θα πρέπει να είναι πιο αξιόπιστες, καθώς η συνάρτηση προσαρμόστηκε ειδικά σε αυτά τα δεδομένα. Το δεύτερο σύνολο τιμών\selectlanguage{english} n ([150:100:1550])\selectlanguage{greek} αντιπροσωπεύει ένα διαφορετικό υποσύνολο για το οποίο γίνονται προβλέψεις. Αυτές οι προβλέψεις αποτελούν προεκτάσεις πέραν του αρχικού εύρους και θα πρέπει να ερμηνεύονται με προσοχή. Η προεκβολή εισάγει αβεβαιότητες και η ακρίβεια των προβλέψεων μπορεί να μειωθεί καθώς απομακρύνεστε από το εύρος που χρησιμοποιήθηκε για την προσαρμογή.

\FloatBarrier % Prevents floats from moving past this point
% Insert your content here
\subsection{}
\begin{figure}[h]
\centering
\includegraphics[width=1\textwidth]{image6.png} 
\label{fig:image7}
\end{figure}
\FloatBarrier % Prevents floats from moving past this point
\selectlanguage{greek}
Αυτός ο κώδικας δημιουργεί ένα υποδιάγραμμα με δύο γραφήματα. Το πρώτο υποδιάγραμμα δείχνει τον έλεγχο ακρίβειας για το\selectlanguage{english} n\selectlanguage{greek} στο [100:100:2000] και το δεύτερο υποδιάγραμμα δείχνει τον έλεγχο ακρίβειας για το\selectlanguage{english} n \selectlanguage{greek} στο [150:100:1550].\\

Για κάθε υποδιάγραμμα:
Οι μπλε δείκτες και γραμμές αντιπροσωπεύουν τους πραγματικούς χρόνους εκτέλεσης.
Η κόκκινη γραμμή αντιπροσωπεύει τους προβλεπόμενους χρόνους εκτέλεσης.
Σύντομη περιγραφή όσων έκανα και σχολιασμός των αποτελεσμάτων:
Υπολογισμός προβλέψεων και πραγματικών χρόνων: Οι προβλεπόμενοι χρόνοι εκτέλεσης υπολογίστηκαν για δύο σύνολα τιμών n χρησιμοποιώντας την κυβική συνάρτηση που λαμβάνεται από το\selectlanguage{english} polyfit. \selectlanguage{greek}Οι πραγματικοί χρόνοι εκτέλεσης μετρήθηκαν για τα ίδια σύνολα τιμών\selectlanguage{english} n.\selectlanguage{greek}
Απεικόνιση: Δημιουργήθηκαν δύο υποδιαγράμματα για την απεικόνιση του ελέγχου ακρίβειας για διαφορετικά εύρη\selectlanguage{english} n\selectlanguage{greek}. Σε κάθε υποδιάγραμμα, οι μπλε δείκτες και γραμμές αντιπροσώπευαν τους πραγματικούς χρόνους εκτέλεσης, ενώ η κόκκινη γραμμή αντιπροσώπευε τους προβλεπόμενους χρόνους εκτέλεσης.\\

Σχολιασμός αποτελεσμάτων:\\

Υποδιάγραμμα 1 \selectlanguage{english} $(n=[100:100:2000])$:
\selectlanguage{greek}
Οι μπλε δείκτες και γραμμές δείχνουν τους πραγματικούς χρόνους εκτέλεσης που προέκυψαν από μετρήσεις. Η κόκκινη γραμμή αντιπροσωπεύει τους προβλεπόμενους χρόνους εκτέλεσης με βάση την κυβική συνάρτηση. Ιδανικά, η κόκκινη γραμμή θα πρέπει να ακολουθεί στενά την τάση των μπλε δεικτών, υποδεικνύοντας ακριβείς προβλέψεις. Τυχόν αποκλίσεις μεταξύ των προβλεπόμενων και των πραγματικών χρόνων μπορούν να παρέχουν πληροφορίες σχετικά με την ακρίβεια του μοντέλου και τους πιθανούς περιορισμούς του.\\

Υποδιάγραμμα 2 \selectlanguage{english } $(n=[150:100:1550])$: \selectlanguage{greek}
Παρόμοια με το πρώτο υποδιάγραμμα, αυτό το υποδιάγραμμα επικεντρώνεται σε ένα διαφορετικό εύρος τιμών\selectlanguage{english} n \selectlanguage{greek}για πρόσθετες δοκιμές. Η σύγκριση μεταξύ των πραγματικών και των προβλεπόμενων χρόνων είναι ζωτικής σημασίας για την αξιολόγηση της αξιοπιστίας των προβλέψεων πέραν του αρχικού συνόλου δεδομένων που χρησιμοποιήθηκε για την προσαρμογή.\\

Παρατηρήσεις:
Η στενή ευθυγράμμιση μεταξύ της κόκκινης γραμμής και των μπλε δεικτών υποδηλώνει καλή ακρίβεια στις προβλέψεις. Τυχόν σημαντικές αποκλίσεις ενδέχεται να υποδεικνύουν περιορισμούς ή σφάλματα στη μοντελοποίηση της κυβικής συνάρτησης. Πρέπει να λάβουμε υπόψη ότι η πρόβλεψη των χρόνων εκτέλεσης για τιμές\selectlanguage{english} n \selectlanguage{greek}πέραν του αρχικού συνόλου δεδομένων περιλαμβάνει παρέκταση και τα αποτελέσματα πρέπει να ερμηνεύονται με προσοχή.

% Insert your content here
\subsection{}
% Insert your content here
\begin{figure}[h]
\centering
\includegraphics[width=1\textwidth]{image7.png} 
\label{fig:image8}
\end{figure}
\FloatBarrier % Prevents floats from moving past this point
\selectlanguage{greek}
Εδώ προσαρμόζει πολυώνυμα βαθμού 2, 3 (κυβικά) και 4 στα δεδομένα του χρόνου εκτέλεσης και συγκρίνει τις προβλέψεις τους. Τα αποτελέσματα απεικονίζονται σε ένα γράφημα όπου οι πραγματικοί χρόνοι εκτέλεσης αναπαρίστανται με μπλε δείκτες και οι προβλέψεις με κόκκινες, πράσινες και μπλε γραμμές για κυβικά, πολυώνυμα 2ου και 4ου βαθμού αντίστοιχα.\\ 

Σύντομη περιγραφή όσων έκανα και σχολιασμός των αποτελεσμάτων:

Παραγωγή δεδομένων: Εκτελέστηκε η λειτουργία αποσύνθεσης \selectlanguage{english}Cholesky \selectlanguage{greek}για πίνακες μεγεθών\selectlanguage{english} $(n = [100:100:2000])$. \selectlanguage{greek}Μετρήθηκαν οι χρόνοι εκτέλεσης για κάθε μέγεθος πίνακα.\\

Προσαρμογή πολυωνύμων: Προσαρμόστηκαν πολυώνυμα βαθμού 2, 3 (κυβικά) και 4 στους μετρούμενους χρόνους εκτέλεσης με τη χρήση της συνάρτησης\selectlanguage{English} polyfit.\selectlanguage{greek} Χρησιμοποίησα βάρη για όλες τις προσαρμογές πολυωνύμων για να ενισχύσει την αξιοπιστία.\\

Υπολογισμός πρόβλεψης: Υπολογισμός προβλεπόμενων χρόνων εκτέλεσης για όλες τις τιμές\selectlanguage{english} n \selectlanguage{greek}χρησιμοποιώντας τους συντελεστές που προέκυψαν για κάθε βαθμό πολυωνύμου.\\

Απεικόνιση αποτελεσμάτων: Δημιουργήθηκε ένα διάγραμμα για τη σύγκριση των πραγματικών χρόνων εκτέλεσης (μπλε δείκτες) με τους προβλεπόμενους χρόνους από τα πολυώνυμα κυβικού (κόκκινη γραμμή), 2ου βαθμού (πράσινη γραμμή) και 4ου βαθμού (μπλε γραμμή).\\

Σχολιασμός των αποτελεσμάτων: Η γραφική παράσταση συγκρίνει οπτικά την ακρίβεια των προβλέψεων που γίνονται από πολυώνυμα διαφορετικού βαθμού. Το κυβικό πολυώνυμο προσδιορίστηκε προηγουμένως ως η καλύτερη προσαρμογή για τα αρχικά δεδομένα στο μέρος 1 της ανάλυσης. Συγκρίνοντας τις προβλέψεις των κυβικών πολυωνύμων, των πολυωνύμων 2ου και 4ου βαθμού, μπορούμε να αξιολογήσουμε πόσο καλά κάθε βαθμός αποτυπώνει την υποκείμενη τάση στους χρόνους εκτέλεσης.\\

Παρατηρήσεις: Οι μπλε δείκτες αντιπροσωπεύουν τους πραγματικούς χρόνους εκτέλεσης, ενώ οι γραμμές αντιπροσωπεύουν τις προβλέψεις. Μια καλή προσαρμογή θα έχει τις προβλεπόμενες γραμμές να ακολουθούν στενά την τάση των πραγματικών σημείων δεδομένων. Οι διαφορές μεταξύ των πραγματικών και των προβλεπόμενων τιμών μπορούν να αναδείξουν τα πλεονεκτήματα και τους περιορισμούς κάθε πολυωνυμικού βαθμού στη μοντελοποίηση των χρόνων εκτέλεσης.

\section{\selectlanguage{greek}Ειδικοί επιλυτές και αραιά μητρώα}
\subsection{}
% Insert your content here
\selectlanguage{english}
\begin{verbatim}
function x = solve_triangular_system(A, B)
    n = size(A, 1);
    x = zeros(n, size(B, 2));
    B_original = B; % Store the original B

    for k = 1:log2(n)
        D_inv = diag(1./diag(A));
        A = D_inv * (A + tril(A, -1) + triu(A, 1));
        B = D_inv * (A + tril(A, -1) + triu(A, 1)) * B_original;
        x = x + D_inv * B;
    end
end

% Example usage:
D = diag([2, 3, 4]);
L = tril(ones(3) - eye(3), -1);
U = triu(ones(3) - eye(3), 1);

A = D - L - U;
B = rand(3, 1);

% Call the function from the other script
x = solve_triangular_system(A, B);

% Display the result
disp(x);
\end{verbatim}

\subsection{}
% Insert your content here
\selectlanguage{greek}
Για να μελετήσουμε τη μέθοδο και τις ιδιότητές της, ας αναλύσουμε τον αλγόριθμο και το αριθμητικό κόστος του ως συνάρτηση του\selectlanguage{english} n\selectlanguage{greek}. Ο αλγόριθμος περιλαμβάνει κυρίως πράξεις μεταξύ διανυσμάτων που αντιπροσωπεύουν τις διαγωνίους των πινάκων. Η βασική πράξη είναι ο πολλαπλασιασμός διανυσμάτων\selectlanguage{english} Hadamard. \selectlanguage{greek}Ας συμβολίσουμε το μέγεθος του συστήματος ως n και τον αριθμό των βημάτων του βρόχου ως \selectlanguage{english}log2(n).\\
\selectlanguage{greek}
Τα κύρια βήματα σε κάθε επανάληψη του βρόχου είναι τα εξής:
Υπολογισμός του\selectlanguage{english} \texttt{D\_inv} \selectlanguage{greek} (διαγώνιο διάνυσμα του αντίστροφου της διαγωνίου του A).
Εκτέλεση πολλαπλασιασμού\selectlanguage{english} Hadamard \selectlanguage{greek}των διανυσμάτων (πολλαπλασιασμός κατά στοιχείο).
Ενημέρωση του πίνακα A και του διανύσματος B.
Ας συμβολίσουμε το μέγεθος των διανυσμάτων που αντιπροσωπεύουν τις διαγωνίους ως \selectlanguage{english} m \selectlanguage{greek} (συνήθως ίσο με n για έναν τετραγωνικό πίνακα). Το αριθμητικό κόστος του πολλαπλασιασμού \selectlanguage{english}Hadamard \selectlanguage{greek}είναι\selectlanguage{english} O(m). \selectlanguage{greek}Σε κάθε επανάληψη του βρόχου, εκτελούμε τον πολλαπλασιασμό \selectlanguage{english}Hadamard\selectlanguage{greek} δύο φορές, μία για την ενημέρωση του A και μία για την ενημέρωση του B.\\
Το συνολικό αριθμητικό κόστος του αλγορίθμου είναι τότε\selectlanguage{english} O(log2(n) * m),\selectlanguage{greek} υποθέτοντας ότι ο πολλαπλασιασμός \selectlanguage{english}Hadamard \selectlanguage{greek}κυριαρχεί στο υπολογιστικό κόστος. Η πολυπλοκότητα του αλγορίθμου κυριαρχείται από τον αριθμό των βημάτων του βρόχου, ο οποίος είναι λογαριθμικός ως προς το μέγεθος του συστήματος.
Αξίζει να σημειωθεί ότι η επιλογή του αρχικού\selectlanguage{english} \texttt{D\_inv}\selectlanguage{greek}περιλαμβάνει τον υπολογισμό του αντίστροφου της διαγωνίου του A. Αυτή η πράξη έχει επίσης αριθμητικό κόστος \selectlanguage{english}O(m). \selectlanguage{greek}Επομένως, το συνολικό αριθμητικό κόστος του αλγορίθμου μπορεί να εκφραστεί ως \selectlanguage{english}O(log2(n) * m + m), \selectlanguage{greek}όπου ο πρώτος όρος οφείλεται στους πολλαπλασιασμούς \selectlanguage{english}Hadamard\selectlanguage{greek} και ο δεύτερος όρος στην αντιστροφή της διαγωνίου.
Στην πράξη, το αριθμητικό κόστος εξαρτάται από τη σπανιότητα και τη δομή των εμπλεκόμενων πινάκων. Εάν οι πίνακες είναι αραιοί, το κόστος του πολλαπλασιασμού\selectlanguage{english} Hadamard \selectlanguage{greek}μπορεί να μειωθεί σημαντικά.\\
Συνοπτικά, το αριθμητικό κόστος του αλγορίθμου είναι περίπου γραμμικό σε σχέση με το μέγεθος των διανυσμάτων που αντιπροσωπεύουν τις διαγωνίους και λογαριθμικό σε σχέση με το μέγεθος του συστήματος \selectlanguage{english}n.

\subsection{}
% Insert your content here
\selectlanguage{greek}
Για να πειραματιστούμε με διαγωνοκυριαρχούμενους πίνακες, μπορούμε να δημιουργήσουμε διαγωνοκυριαρχούμενους τριγωνικούς πίνακες και να παρατηρήσουμε τη συμπεριφορά των υπερ- και υποδιαγωνίων τους μέσω των επαναλήψεων του αλγορίθμου. Συγκεκριμένα, μπορούμε να εξετάσουμε τις νόρμες αυτών των διαγωνίων για να δούμε πώς εξελίσσονται κατά τη διάρκεια της διαδικασίας.
\begin{figure}[h]
\centering
\includegraphics[width=0.8\textwidth]{image8.png} 
\label{fig:image9}
\end{figure}
\begin{figure}[h]
\centering
\includegraphics[width=0.8\textwidth]{image9.png} 
\label{fig:image10}
\end{figure}
\begin{figure}[h]
\centering
\includegraphics[width=0.8\textwidth]{image10.png} 
\label{fig:image11}
\end{figure}
\begin{figure}[h]
\centering
\includegraphics[width=0.8\textwidth]{image11.png} 
\label{fig:image12}
\end{figure}
\begin{figure}[h]
\centering
\includegraphics[width=0.8\textwidth]{image12.png} 
\label{fig:image13}
\end{figure}
\FloatBarrier % Prevents floats from moving past this point

\section{\selectlanguage{greek}Τανυστές}
\subsection{}
% Insert your content here
\begin{figure}[h]
\centering
\includegraphics[width=0.8\textwidth]{image13.png} 
\label{fig:image14}
\end{figure}
\FloatBarrier % Prevents floats from moving past this point
\subsection{}
% Insert your content here
\selectlanguage{english}
  \texttt{ttv\_myid.m}\\

\selectlanguage{greek}Αυτή η συνάρτηση εκτελεί τροπικό πολλαπλασιασμό ενός πολυδιάστατου πίνακα \selectlanguage{english}X \selectlanguage{greek}με ένα διάνυσμα στήλης\selectlanguage{english} V \selectlanguage{greek} κατά μήκος της καθορισμένης διάστασης N. Οι τύποι εισόδου και οι διαστάσεις ελέγχονται για να διασφαλιστεί ότι ο \selectlanguage{english} X \selectlanguage{greek} είναι πίνακας και το\selectlanguage{english} V \selectlanguage{greek} διάνυσμα στήλης. Επιπλέον, οι διαστάσεις ελέγχονται για συμβατότητα με τη λειτουργία τροπικού πολλαπλασιασμού.\\
\selectlanguage{english} 

\texttt{ttm\_myid.m} \selectlanguage{greek}\\

Παρόμοια με την \selectlanguage{english}\texttt{ttv\_myid.m},\selectlanguage{greek} αυτή η συνάρτηση εκτελεί τροπικό πολλαπλασιασμό, αλλά με μια μικρή διαφοροποίηση στη λειτουργία. Πολλαπλασιάζει τον πολυδιάστατο πίνακα \selectlanguage{english}X \selectlanguage{greek} με ένα διάνυσμα στήλης\selectlanguage{english} V\selectlanguage{greek} κατά μήκος της καθορισμένης διάστασης\selectlanguage{english} N.\selectlanguage{greek} Οι τύποι και οι διαστάσεις εισόδου ελέγχονται για να διασφαλιστεί ότι ο \selectlanguage{english}X \selectlanguage{greek}είναι πίνακας και το \selectlanguage{english}V \selectlanguage{greek}διάνυσμα στήλης. Οι διαστάσεις επαληθεύονται επίσης για συμβατότητα με την πράξη τροπικού πολλαπλασιασμού.\\
\selectlanguage{english}

\texttt{ttt\_myid.m} \selectlanguage{greek}\\

Αυτή η συνάρτηση παρέχει μια απλή μορφή εξωτερικών και εσωτερικών παραγώγων τανυστών. Για το εξωτερικό γινόμενο, επιστρέφει ένα \selectlanguage{english} MDA\selectlanguage{greek} που περιέχει τις τιμές και τη σωστή διάταξη του τανυστικού γινομένου των A και B. Για το εσωτερικό γινόμενο, επιστρέφει το κλιμακωτό αποτέλεσμα του στοιχειομετρικού πολλαπλασιασμού και του αθροίσματος των A και B. Οι τύποι εισόδου ελέγχονται για να διασφαλιστεί ότι τόσο ο A όσο και ο B είναι πίνακες. Το προαιρετικό όρισμα mode επιτρέπει στο χρήστη να καθορίσει αν επιθυμεί το εξωτερικό ή το εσωτερικό γινόμενο.\\
\selectlanguage{english}

\texttt{test\_tensor.m} \selectlanguage{greek}\\

Αυτή η συνάρτηση αρχικοποιεί τους τανυστές A και B, μαζί με άλλες παραμέτρους, και στη συνέχεια ελέγχει την ορθότητα των υλοποιημένων συναρτήσεων \selectlanguage{english}(\texttt{ttv\_myid.m}, \texttt{ttm\_myid.m}, \texttt{ttt\_myid.m})\selectlanguage{greek} συγκρίνοντας τα αποτελέσματά τους με αντίστοιχες λειτουργίες στην εργαλειοθήκη\selectlanguage{english} Tensor Toolbox.\selectlanguage{greek} H συνάρτηση ελέγχει για τυχόν σφάλματα ή αναντιστοιχίες και αναφέρει τα αποτελέσματα. Στόχος είναι να διασφαλιστεί ότι οι προσαρμοσμένες υλοποιήσεις παράγουν τα ίδια αποτελέσματα με τις εντολές της \selectlanguage{english}Tensor Toolbox.


\section{\selectlanguage{greek}Επίλυση ΣΘΟ συστημάτων με\selectlanguage{english} PCG}
\selectlanguage{greek}
Στην συγκεκριμένη ερώτηση δεν μπόρεσα να βγάλω σωστά αποτελέσματα, παρόλα αυτά επειδή ασχολήθηκα πολύ έφτιαξα αρκετά αξιόλογο κώδικα για να μπορώ να εξηγήσω το σκεπτικό πίσω από τον κώδικα.
\selectlanguage{english}
\subsection{poison\_test\_1070263.m}
% Insert your content here
\selectlanguage{greek}
Ρύθμιση παραμέτρων: Όρισα παραμέτρους όπως ο αριθμός των εσωτερικών κόμβων\selectlanguage{english} (n), \selectlanguage{greek}ο αριθμός των οριακών συνθηκών\selectlanguage{english} (m)\selectlanguage{greek} και η απόσταση μεταξύ διαδοχικών κόμβων\selectlanguage{english} (h).\\
\selectlanguage{greek}
Ορισμός του πλέγματος: Δημιούργησα ένα πλέγμα \selectlanguage{english}(X and Y) \selectlanguage{greek}χρησιμοποιώντας το \selectlanguage{english}meshgrid \selectlanguage{greek}για να αναπαραστήσω το χωρικό πεδίο.
Κατασκεύασα τον πίνακα\selectlanguage{english} Poisson \selectlanguage{greek}και το διάνυσμα της δεξιάς πλευράς: Ύστερα χρησιμοποιήσα μια συνάρτηση \selectlanguage{english} texttt{constructPoissonMatrix\_1070263} \selectlanguage{greek} για να κατασκευάσω τον πίνακα \selectlanguage{english}Poisson A\selectlanguage{greek} και να δημιουργήσω το διάνυσμα\selectlanguage{english} b\selectlanguage{greek} της δεξιάς πλευράς.\\

Ρύθμιση δοκιμών προαπαιτούμενων συνθηκών\selectlanguage{english} (Preconditioning Tests):\selectlanguage{greek}
Δοκίμασα τον επαναληπτικό επιλύτη (μέθοδος συζυγών βαθμίδων με προϋποθέσεις) χωρίς προϋποθέσεις \selectlanguage{english}("none").\selectlanguage{greek} Επίσης έκανα δοκιμή με ελλιπή προρύθμιση \selectlanguage{english} Cholesky ('ichol')\selectlanguage{greek} και δοκιμή με προσαρμοσμένη προεπεξεργασία \selectlanguage{english}('custom').\\
\selectlanguage{greek}\\
Απεικόνιση:
Σχεδίασα το σχετικό υπόλειμμα για κάθε επανάληψη και για τις τρεις στρατηγικές προετοιμασίας. Έκαν απεικόνιση του σχετικού σφάλματος για κάθε επανάληψη και για τις τρεις στρατηγικές προετοιμασίας.Τέλος, γίνεται η εμφάνιση των παραγόμενων γραφικών παραστάσεων.

\subsection{\selectlanguage{english}pcg\_1070263}
% Insert your content here
\selectlanguage{greek}
Η συνάρτηση του MATLAB είναι μια τροποποιημένη έκδοση της μεθόδου Preconditioned Conjugate Gradients\selectlanguage{english} (PCG),\selectlanguage{greek} με την ονομασία\selectlanguage{english} texttt{pcg\_1070263}.\selectlanguage{greek} Έχει σχεδιαστεί για την επίλυση ενός συστήματος γραμμικών εξισώσεων\selectlanguage{english} $Ax=b$ \selectlanguage{greek}με τη δυνατότητα χρήσης ενός προρυθμιστή.\\

Παράμετροι εισόδου:\\
A: Ο πίνακας συντελεστών ή μια λαβή συνάρτησης που αναπαριστά έναν γραμμικό τελεστή.\\
\selectlanguage{english}b:\selectlanguage{greek} Το διάνυσμα της δεξιάς πλευράς.
\selectlanguage{english}tol: \selectlanguage{greek}Ανοχή σύγκλισης.
\selectlanguage{english}maxit: \selectlanguage{greek}Μέγιστος αριθμός επαναλήψεων.\\
\selectlanguage{english}preconditioner:\selectlanguage{greek} Συμβολοσειρά που καθορίζει τον τύπο του\selectlanguage{english} preconditioner\selectlanguage{greek} που θα χρησιμοποιηθεί
\selectlanguage{english}('none', 'ichol', ή 'custom').\selectlanguage{greek}\\
\selectlanguage{english}$x0$: \selectlanguage{greek}Αρχική εικασία για τη λύση.\\
\selectlanguage{english}xsol:\selectlanguage{greek} Ακριβής λύση (για τον υπολογισμό σφαλμάτων).\selectlanguage{english}\\
varargin:\selectlanguage{greek} Πρόσθετα ορίσματα που μεταβιβάζονται στις συναρτήσεις του γινόμενου μήτρας-διανύσματος και του\selectlanguage{english} preconditioner.\selectlanguage{greek}\\

Παράμετροι εξόδου:\\
\selectlanguage{english}x:\selectlanguage{greek} Η υπολογισμένη λύση.\\
\selectlanguage{english}flag: \selectlanguage{greek}Σημαία σύγκλισης (0 για σύγκλιση).\\
\selectlanguage{english}relres:\selectlanguage{greek} Σχετικό υπόλειμμα.\\
\selectlanguage{english}iter: \selectlanguage{greek}Αριθμός επαναλήψεων που εκτελούνται.\\
\selectlanguage{english}resvec: \selectlanguage{greek}Διάνυσμα που αποθηκεύει τη νόρμα του υπολοίπου σε κάθε επανάληψη.\\
\selectlanguage{english}errvec: \selectlanguage{greek}Διάνυσμα που αποθηκεύει το σχετικό σφάλμα σε κάθε επανάληψη.\\

Προσδιορισμός τύπου λειτουργίας:\\
Η συνάρτηση προσδιορίζει αν ο πίνακας συντελεστών A είναι ένας αριθμητικός πίνακας ή μια συνάρτηση, ελέγχοντας τον τύπο του (πίνακας ή λαβή συνάρτησης).
Ορίζει ανάλογα μια κατάλληλη λαβή συνάρτησης\selectlanguage{english} (afun).\\

\selectlanguage{greek}
Επικύρωση παραμέτρων εισόδου:\\
Ελέγχει τα μεγέθη και τους τύπους των παραμέτρων εισόδου, όπως \selectlanguage{english}A, b, tol, maxit, preconditioner, x0 xsol.\\\selectlanguage{greek}
Παρέχει προεπιλεγμένες τιμές για ορισμένες παραμέτρους εάν δεν έχουν καθοριστεί.\\

Εφαρμογή προαπαιτούμενου:\\
Εάν έχει καθοριστεί ένας προκλιματιστής\selectlanguage{english} ('ichol' ή 'custom'), \selectlanguage{greek}εφαρμόζει την αντίστοιχη λειτουργία προκλιματισμού.
Εάν δεν έχει καθοριστεί κανένας προκαθοριστής, χρησιμοποιεί τον πίνακα ταυτότητας ως προκαθοριστή τοποθέτησης.\\

Κύριος επαναληπτικός βρόχος:\\
Εφαρμόζει τη μέθοδο\selectlanguage{english} Preconditioned Conjugate Gradients,\selectlanguage{greek} επαναλαμβάνοντας μέχρι τη σύγκλιση ή τη συμπλήρωση του μέγιστου αριθμού επαναλήψεων.
Υπολογίζει την κατεύθυνση αναζήτησης \selectlanguage{english}(p) \selectlanguage{greek}και το γινόμενο μήτρας-διανύσματος\selectlanguage{english} (q) \selectlanguage{greek}σε κάθε επανάληψη.\\

Έλεγχοι σύγκλισης και στασιμότητας:\\
Παρακολουθεί τη σύγκλιση συγκρίνοντας τη νόρμα του υπολείμματος \selectlanguage{english}texttt{(normr\_act)} \selectlanguage{greek}με την καθορισμένη ανοχή \selectlanguage{english}(tol).\selectlanguage{greek}
Ελέγχει τη στασιμότητα εξετάζοντας το λόγο της νόρμας της κατεύθυνσης αναζήτησης προς τη νόρμα της λύσης.\\

Αποθήκευση αποτελεσμάτων:\\
Καταγράφει τις σχετικές πληροφορίες, όπως τις υπολειμματικές νόρμες σε κάθε επανάληψη \selectlanguage{english}(resvec),\selectlanguage{greek} το σφάλμα σε κάθε επανάληψη\selectlanguage{english} (errvec) \selectlanguage{greek}και τον αριθμό των επαναλήψεων που εκτελέστηκαν.\\

Χειρισμός ακραίων περιπτώσεων:\\
Αντιμετωπίζει ειδικές περιπτώσεις, όπως μια αρχική εικασία που είναι ήδη μια καλή λύση ή ένα διάνυσμα της δεξιάς πλευράς με όλα τα μηδενικά.\\

Έξοδος και εμφάνιση:\\
Επιστρέφει την υπολογισμένη λύση \selectlanguage{english}(x), \selectlanguage{greek}τη σημαία σύγκλισης \selectlanguage{english}(flag),\selectlanguage{greek} το σχετικό υπόλοιπο (relres), τον αριθμό των επαναλήψεων \selectlanguage{english}(iter) \selectlanguage{greek}και τα διανύσματα που αποθηκεύουν τις πληροφορίες για το υπόλοιπο και το σφάλμα.
Εμφανίζει πληροφορίες σύγκλισης εάν δεν ζητείται ρητά η σημαία εξόδου.\\

Χειρισμός σφαλμάτων και προειδοποιήσεων:\\
Ελέγχει για πιθανά προβλήματα κατά τη διάρκεια των επαναλήψεων, όπως άπειρες τιμές, και θέτει τις κατάλληλες σημαίες \selectlanguage{english}(flag)\selectlanguage{greek} και προειδοποιήσεις.\\

Ευελιξία:\\
Επιτρέπει στους χρήστες να ορίζουν προσαρμοσμένες στρατηγικές προκλιμάκωσης παρέχοντας τη δική τους συνάρτηση προκλιμάκωσης.
Αυτή η συνάρτηση έχει σχεδιαστεί για να είναι ευέλικτη, να χειρίζεται διαφορετικούς τύπους πινάκων εισόδου, να παρέχει επιλογές για την προαποτύπωση και να προσφέρει ευελιξία στο χειρισμό διαφόρων σεναρίων σύγκλισης. Στόχος της είναι να είναι ισχυρή στην επίλυση γραμμικών συστημάτων με τη μέθοδο \selectlanguage{english}Preconditioned Conjugate Gradients.

\subsection{constructPoissonMatrix\_1070263}
% Insert your content here
\selectlanguage{greek}
Αυτή η συνάρτηση παράγει τον πίνακα \selectlanguage{english}Poisson\selectlanguage{greek} για δεδομένο πλέγμα και οριακές συνθήκες. \\

Πιο αναλυτικά:\\
Παράμετροι εισόδου:\\
\selectlanguage{english}X:\selectlanguage{greek} Συντεταγμένες πλέγματος.\\
texttt{~} (αχρησιμοποίητο): Παράμετρος που αγνοείται.\\
\selectlanguage{english}h:\selectlanguage{greek} Απόσταση μεταξύ διαδοχικών κόμβων.\\

Κατασκευή πινάκων:\\
Αρχικοποιεί έναν κενό αραιό πίνακα A με διαστάσεις n×m όπου n και m είναι οι διαστάσεις του πλέγματος \selectlanguage{english}X.\selectlanguage{greek} Επαναλαμβάνει κάθε σημείο του πλέγματος και κατασκευάζει τις καταχωρήσεις του πίνακα \selectlanguage{english}Poisson.\selectlanguage{greek}
Ορίζει το grid point σε\selectlanguage{english} $-4/h2$\selectlanguage{greek}
Θέτει τα στοιχεία εκτός διαγωνίου που αντιστοιχούν στα γειτονικά σημεία σε \selectlanguage{english}$1/h2$\\\selectlanguage{greek}

Χειρισμός οριακών συνθηκών:\\
Μηδενίζει τα στοιχεία της πρώτης και της τελευταίας γραμμής (που αντιπροσωπεύουν τα άνω και κάτω όρια) και της πρώτης και της τελευταίας στήλης (που αντιπροσωπεύουν τα αριστερά και τα δεξιά όρια). Αυτό εφαρμόζει αποτελεσματικά τις συνοριακές συνθήκες\selectlanguage{english} Dirichlet, \selectlanguage{greek}καθορίζοντας τις τιμές της λύσης στα συνοριακά σημεία στο μηδέν.\\

Έξοδος:\\
Επιστρέφει τον κατασκευασμένο πίνακα \selectlanguage{english}Poisson A.\selectlanguage{greek}
Αυτή η συνάρτηση είναι ειδικά προσαρμοσμένη για την κατασκευή του πίνακα Poisson για ένα δισδιάστατο πλέγμα με οριακές συνθήκες\selectlanguage{english} Dirichlet.\selectlanguage{greek}

\subsection{\selectlanguage{english}customPreconditioner}
% Insert your content here
H συνάρτηση, παράγει έναν προσαρμοσμένο ατελή προρυθμιστή \selectlanguage{english}Cholesky (IC) \selectlanguage{greek}για έναν δεδομένο πίνακα A χρησιμοποιώντας τη συνάρτηση\selectlanguage{english} ichol.\selectlanguage{greek}\\

Παράμετρος εισόδου:\\
A: Ο πίνακας για τον οποίο κατασκευάζεται ο \selectlanguage{english}preconditioner.\selectlanguage{greek}\\
Κατασκευή\selectlanguage{english} preconditioner:\selectlanguage{greek} Ο πίνακας που χρησιμοποιείται για την κατασκευή του προκοντίσιονερ χρησιμοποιεί τη συνάρτηση \selectlanguage{english}ichol,\selectlanguage{greek} η οποία υπολογίζει μια ατελή παραγοντοποίηση \selectlanguage{english}Cholesky\selectlanguage{greek} του πίνακα εισόδου.\\

Οι επιλογές για την\selectlanguage{english} ichol\selectlanguage{greek} καθορίζονται μέσω μιας δομής \selectlanguage{english}struct:\selectlanguage{greek}\\
\selectlanguage{english}'type':\selectlanguage{greek} Καθορίζει τον τύπο της ατελούς παραγοντοποίησης \selectlanguage{english}('ict' \selectlanguage{greek}για \selectlanguage{english}IC\selectlanguage{greek} με κατώφλι).\\
\selectlanguage{english}'droptol': \selectlanguage{greek}Ορίζει την ανοχή πτώσης για μικρά στοιχεία κατά την παραγοντοποίηση.\\
\selectlanguage{english}'shape':\selectlanguage{greek} Καθορίζει εάν θα υπολογιστεί το άνω ή το κάτω τριγωνικό μέρος της παραγοντοποίησης.\\

\subsection{\selectlanguage{english}incompleteCholesky}
\selectlanguage{greek}
% Insert your content here
Παράμετρος εισόδου:\\
A: Ο πίνακας για τον οποίο εκτελείται η ατελής παραγοντοποίηση\selectlanguage{english} Cholesky.\selectlanguage{greek}\\

Παραγοντοποίηση:\\
Χρησιμοποιεί τη συνάρτηση \selectlanguage{english}cholinc \selectlanguage{greek}για την εκτέλεση ατελούς παραγοντοποίησης\selectlanguage{english} Cholesky \selectlanguage{greek}με μηδενική πλήρωση ("0").
Η παραγοντοποίηση εκτελείται με επίπεδο μηδενικής πλήρωσης, πράγμα που σημαίνει ότι δεν προστίθενται πρόσθετα στοιχεία στον παραγοντοποιημένο πίνακα για να διασφαλιστεί η αραιότητα.\\

Έξοδος:\\
Επιστρέφει τον κατώτερο τριγωνικό παράγοντα\selectlanguage{english} L\selectlanguage{greek} και τον ανώτερο τριγωνικό παράγοντα \selectlanguage{english}U\selectlanguage{greek} της ατελούς παραγοντοποίησης \selectlanguage{english}Cholesky.\selectlanguage{greek}

\subsection{\selectlanguage{english}solveExactPoisson}
% Insert your content here
\selectlanguage{greek}
Yπολογίζει την ακριβή λύση για την εξίσωση Poisson σε ένα δεδομένο πλέγμα που ορίζεται από τα\selectlanguage{english} X \selectlanguage{greek}και\selectlanguage{english} Y\selectlanguage{greek}
\subsection{\selectlanguage{english}SuiteSparse.m}
% Insert your content here
\selectlanguage{greek}
Αυτός ο κώδικας επιλύει ένα γραμμικό σύστημα που ορίζεται από Τo μητρώο υπ’ αριθμ. 1 του \selectlanguage{english}SuiteSparse,\selectlanguage{greek} δηλ. το texttt{1138\_bus} και το διάνυσμα της δεξιάς πλευράς\selectlanguage{english} b \selectlanguage{greek}χρησιμοποιώντας τη μέθοδο\selectlanguage{english} Preconditioned Conjugate Gradients (PCG) \selectlanguage{greek}με την τροποποιημένη συνάρτηση\selectlanguage{english} texttt{(pcg\_1070263)}. \\
\selectlanguage{greek}
Ορισμός μητώου:\\
A = Πρόβλημα: Αρχικοποιεί τον πίνακα συντελεστών A με βάση το μητρώο\selectlanguage{english} Problem. \selectlanguage{greek}\\

Διάνυσμα δεξιάς πλευράς:\\
\selectlanguage{english}b = rand(size(A, 1), 1);: \selectlanguage{greek}Δημιουργεί ένα τυχαίο διάνυσμα δεξιάς πλευράς b με τον ίδιο αριθμό γραμμών με τον πίνακα A.\\

Προκαθοριστής και όριο επανάληψης:\\
\selectlanguage{english}preconditioner = 'ichol';: \selectlanguage{greek}Καθορίζει τον τύπο του \selectlanguage{english}preconditioner\selectlanguage{greek} που θα χρησιμοποιηθεί στη μέθοδο PCG. Σε αυτή την περίπτωση, ορίζεται σε ατελή παραγοντοποίηση \selectlanguage{english}Cholesky ('ichol').\selectlanguage{greek} Μπορείτε να το αλλάξετε σε άλλες επιλογές όπως \selectlanguage{english}'none', 'custom',\selectlanguage{greek} κ.λπ. \\
\selectlanguage{english}maxiter = 4 * size(A, 1);: \selectlanguage{greek}Ορίζει ένα ανώτερο όριο στον αριθμό των επαναλήψεων για τη μέθοδο \selectlanguage{english}PCG.\\

PCG Solver:\\
[x, flag, relres, iter, resvec, errvec] = \texttt{pcg\_myid\_1070263}(A, b, 1e-6, maxiter, preconditioner, [], [], 'resvec', [], 'errvec', []);:\selectlanguage{greek}  Καλεί τον τροποποιημένο επιλυτή\selectlanguage{english} PCG \texttt{(pcg\_myid\_1070263)} \selectlanguage{greek}για την επίλυση του γραμμικού συστήματος. Επιστρέφει τη λύση \selectlanguage{english}x,\selectlanguage{greek} τη σημαία της σημαίας σύγκλισης, το σχετικό υπόλοιπο\selectlanguage{english} relres,\selectlanguage{greek} τον αριθμό επαναλήψεων \selectlanguage{english}iter\selectlanguage{greek} και τα διανύσματα που αποθηκεύουν τις πληροφορίες για το υπόλοιπο \selectlanguage{english}(resvec) \selectlanguage{greek}και το σφάλμα\selectlanguage{english} (errvec).\selectlanguage{greek}

Αυτός ο κώδικας συγκρίνει την απόδοση της μεθόδου \selectlanguage{english}(PCG) \selectlanguage{greek}με την τροποποιημένη συνάρτηση \selectlanguage{english}\texttt{(pcg\_1070263)}.\selectlanguage{greek}Για την επίλυση ενός γραμμικού συστήματος χρησιμοποιώντας διαφορετικές στρατηγικές προετοιμασίας.\\

Αριθμός διαδρομών:\\
Αριθμός διαδρομών = 10;: Καθορίζει τον αριθμό των εκτελέσεων για τη συγκριτική αξιολόγηση.
Συγκριτική αξιολόγηση χωρίς προαπαιτούμενα:
Χρησιμοποιεί έναν βρόχο για την εκτέλεση της μεθόδου\selectlanguage{english} PCG \selectlanguage{greek}χωρίς προαπαιτούμενα για πολλαπλές εκτελέσεις.
Μετρά το χρόνο εκτέλεσης για κάθε εκτέλεση και υπολογίζει το μέσο χρόνο.\\

\selectlanguage{english}Benchmarking With Incomplete Cholesky Preconditioning:\\
\selectlanguage{greek}Παρόμοια με την περίπτωση χωρίς προαπαιτούμενα, αλλά με τον προαπαιτούμενο\selectlanguage{english} 'ichol'.\selectlanguage{greek}
Μετρά το χρόνο εκτέλεσης για κάθε εκτέλεση και υπολογίζει το μέσο χρόνο.
Συγκριτική αξιολόγηση με προσαρμοσμένη προεπεξεργασία:
Παρόμοια με τις προηγούμενες περιπτώσεις, αλλά με έναν προσαρμοσμένο \selectlanguage{english}preconditioner.\selectlanguage{greek}
Μετρά το χρόνο εκτέλεσης για κάθε εκτέλεση και υπολογίζει το μέσο χρόνο.\\

Εμφάνιση του μέσου χρόνου εκτέλεσης:\\
Εκτυπώνει τους μέσους χρόνους εκτέλεσης για κάθε στρατηγική προκλιμάκωσης.\\

Αρχικοποίηση πίνακα και συλλογή δεδομένων:\\
Αρχικοποιεί πίνακες για την αποθήκευση των αποτελεσμάτων για κάθε στρατηγική προκλιμάκωσης.
Συμπληρώνει τους πίνακες με πληροφορίες όπως η μέθοδος που χρησιμοποιήθηκε, η σημαία σύγκλισης, το τελικό σχετικό υπόλειμμα, το τελικό σχετικό σφάλμα και ο μέσος χρόνος εκτέλεσης.\\

Εμφάνιση πινάκων:\\
Εμφανίζει τους πίνακες που περιέχουν τα αποτελέσματα για κάθε στρατηγική προκλιμάκωσης.\\

\begin{center}
\textbf{Ευχαριστώ για τον χρόνο σας!}
\end{center}

% End of the document
\end{document}
